\documentclass{article}
\usepackage{lmodern}
\usepackage{amsmath}
\usepackage{listings}
\usepackage{fullpage}
\usepackage{parskip}
\usepackage{xcolor}
\lstset{
	basicstyle=\ttfamily,
	columns=fullflexible,
	frame=single,
	breaklines=true,
	postbreak=\mbox{\textcolor{red}{$\hookrightarrow$}\space},
}
\begin{document}
\title{Experiment `p6\_6\_alternating\_sum' Results}
\author{\today}
\date{}
\maketitle
\textbf{Experiment outcome: }SUCCESS
\\\textbf{Bad responses: }0
\\\textbf{Responses containing}~\texttt{assume}~\textbf{: }0
\\\textbf{Resolution attempts: }6
\\\textbf{Hard fails (resolution): }0
\\\textbf{Soft fails (resolution): }0
\\\textbf{Verification attempts: }6
\section*{Problem Specification}
\textbf{Problem name: }p6\_6\_alternating\_sum
\\\textbf{Natural language statement: }Compute the alternating sum of all elements in an array. For example, if your method reads the input [1, 4, 9, 16, 9, 7, 4, 9, 11] then it computes 1 – 4 + 9 – 16 + 9 – 7 + 4 – 9 + 11 = –2.
\\\textbf{Method signature: }p6\_6\_alternating\_sum(arr: seq<int>) returns (alt\_sum: int)
\subsection*{Ensures}
\begin{itemize}
\item \texttt{alt\_sum == sum\_arrray(arr, 1)}
\end{itemize}
\subsection*{Functional Code Given}
\begin{lstlisting}
function sum_arrray(arr : seq<int>, i : int) : int
{
  if |arr| == 0 then 0
  else arr[0] * i + sum_arrray(arr[1..], i * -1)
}
\end{lstlisting}
\clearpage
\section*{GenAI interactions}
Below you will find all interactions between the `user' (program) and the `assistant' (OpenAI).
\subsection*{Program $\rightarrow$ GenAI}
\begin{lstlisting}
You are given the following task to perform in Dafny:

Compute the alternating sum of all elements in an array. For example, if your method reads the input [1, 4, 9, 16, 9, 7, 4, 9, 11] then it computes 1 – 4 + 9 – 16 + 9 – 7 + 4 – 9 + 11 = –2.

The signature should be:

method p6_6_alternating_sum(arr: seq<int>) returns (alt_sum: int)

The method should respect the following contract:

ensures alt_sum == sum_arrray(arr, 1)

The contract uses the following dafny code:

function sum_arrray(arr : seq<int>, i : int) : int
{
  if |arr| == 0 then 0
  else arr[0] * i + sum_arrray(arr[1..], i * -1)
}

These function(s) must not be used in your implementation of the method.

Produce and show only the Dafny body of this method, including the curly braces that surround it. Do not show the signature nor contract. You must not use 'assume' anywhere in your code.

\end{lstlisting}
\subsection*{GenAI $\rightarrow$ Program}
\textbf{System fingerprint: }
\\\textbf{ID: }chatcmpl-Co5AyB7gTHX0mcbzmtQemu1IDWvXP
\\\textbf{Model: }gpt-5.1-chat-latest
\\\textbf{Created at: }1766052332
\\\textbf{Finish reason: }stop
\\\textbf{Usage: }548 tokens in, and 341 tokens out
\begin{lstlisting}
{
  var i := 0;
  var sign := 1;
  alt_sum := 0;
  while i < |arr|
    invariant 0 <= i <= |arr|
    invariant alt_sum == sum_arrray(arr[..i], 1)
    invariant (sign == 1 && i % 2 == 0) || (sign == -1 && i % 2 == 1)
  {
    alt_sum := alt_sum + arr[i] * sign;
    sign := sign * -1;
    i := i + 1;
  }
}
\end{lstlisting}
\subsection*{Program $\rightarrow$ GenAI}
\begin{lstlisting}
When using dafny verify, the below error is emitted and verify fails:

problem_data/p6_6_alternating_sum/program.dfy(9,0): Error: a postcondition could not be proved on this return path
  |
9 | {
  | ^

problem_data/p6_6_alternating_sum/program.dfy(8,17): Related location: this is the postcondition that could not be proved
  |
8 | 	ensures alt_sum == sum_arrray(arr, 1)
  |                  ^^

problem_data/p6_6_alternating_sum/program.dfy(15,22): Error: this invariant could not be proved to be maintained by the loop
 Related message: loop invariant violation
   |
15 |     invariant alt_sum == sum_arrray(arr[..i], 1)
   |                       ^^


Dafny program verifier finished with 1 verified, 2 errors

Correct the error by altering only the method body. Produce and show only the Dafny body, including the curly braces that surround it. Do not show the signature nor contract. You must not use 'assume' anywhere in your code.

\end{lstlisting}
\subsection*{GenAI $\rightarrow$ Program}
\textbf{System fingerprint: }
\\\textbf{ID: }chatcmpl-Co5B6eiIIitQittZEyJ8Vi4KFrqic
\\\textbf{Model: }gpt-5.1-chat-latest
\\\textbf{Created at: }1766052340
\\\textbf{Finish reason: }stop
\\\textbf{Usage: }918 tokens in, and 120 tokens out
\begin{lstlisting}
{
  alt_sum := 0;
  var i := 0;
  while i < |arr|
    invariant 0 <= i <= |arr|
    invariant alt_sum == sum_arrray(arr[..i], 1)
  {
    var sign := if i % 2 == 0 then 1 else -1;
    alt_sum := alt_sum + arr[i] * sign;
    i := i + 1;
  }
}
\end{lstlisting}
\subsection*{Program $\rightarrow$ GenAI}
\begin{lstlisting}
When using dafny verify, the below error is emitted and verify fails:

problem_data/p6_6_alternating_sum/program.dfy(9,0): Error: a postcondition could not be proved on this return path
  |
9 | {
  | ^

problem_data/p6_6_alternating_sum/program.dfy(8,17): Related location: this is the postcondition that could not be proved
  |
8 | 	ensures alt_sum == sum_arrray(arr, 1)
  |                  ^^

problem_data/p6_6_alternating_sum/program.dfy(14,22): Error: this invariant could not be proved to be maintained by the loop
 Related message: loop invariant violation
   |
14 |     invariant alt_sum == sum_arrray(arr[..i], 1)
   |                       ^^


Dafny program verifier finished with 1 verified, 2 errors

Correct the error by altering only the method body. Produce and show only the Dafny body, including the curly braces that surround it. Do not show the signature nor contract. You must not use 'assume' anywhere in your code.

\end{lstlisting}
\subsection*{GenAI $\rightarrow$ Program}
\textbf{System fingerprint: }
\\\textbf{ID: }chatcmpl-Co5B9lmXq4cH88Dn4cVVLaBVRLwN5
\\\textbf{Model: }gpt-5.1-chat-latest
\\\textbf{Created at: }1766052343
\\\textbf{Finish reason: }stop
\\\textbf{Usage: }1277 tokens in, and 458 tokens out
\begin{lstlisting}
{
  alt_sum := 0;
  var i := 0;
  var sign := 1;
  while i < |arr|
    invariant 0 <= i <= |arr|
    invariant alt_sum == sum_arrray(arr[..i], 1)
    invariant sign == (if i % 2 == 0 then 1 else -1)
  {
    alt_sum := alt_sum + arr[i] * sign;
    sign := sign * -1;
    i := i + 1;
  }
}
\end{lstlisting}
\subsection*{Program $\rightarrow$ GenAI}
\begin{lstlisting}
When using dafny verify, the below error is emitted and verify fails:

problem_data/p6_6_alternating_sum/program.dfy(9,0): Error: a postcondition could not be proved on this return path
  |
9 | {
  | ^

problem_data/p6_6_alternating_sum/program.dfy(8,17): Related location: this is the postcondition that could not be proved
  |
8 | 	ensures alt_sum == sum_arrray(arr, 1)
  |                  ^^

problem_data/p6_6_alternating_sum/program.dfy(15,22): Error: this invariant could not be proved to be maintained by the loop
 Related message: loop invariant violation
   |
15 |     invariant alt_sum == sum_arrray(arr[..i], 1)
   |                       ^^


Dafny program verifier finished with 1 verified, 2 errors

Correct the error by altering only the method body. Produce and show only the Dafny body, including the curly braces that surround it. Do not show the signature nor contract. You must not use 'assume' anywhere in your code.

\end{lstlisting}
\subsection*{GenAI $\rightarrow$ Program}
\textbf{System fingerprint: }
\\\textbf{ID: }chatcmpl-Co5BGlAiEqTupRNyctMw16g1khhhY
\\\textbf{Model: }gpt-5.1-chat-latest
\\\textbf{Created at: }1766052350
\\\textbf{Finish reason: }stop
\\\textbf{Usage: }1654 tokens in, and 248 tokens out
\begin{lstlisting}
{
  alt_sum := 0;
  var i := 0;
  while i < |arr|
    invariant 0 <= i <= |arr|
    invariant alt_sum == sum_arrray(arr[..i], 1)
  {
    var sign := if i % 2 == 0 then 1 else -1;
    alt_sum := alt_sum + arr[i] * sign;
    i := i + 1;
  }
}
\end{lstlisting}
\subsection*{Program $\rightarrow$ GenAI}
\begin{lstlisting}
When using dafny verify, the below error is emitted and verify fails:

problem_data/p6_6_alternating_sum/program.dfy(9,0): Error: a postcondition could not be proved on this return path
  |
9 | {
  | ^

problem_data/p6_6_alternating_sum/program.dfy(8,17): Related location: this is the postcondition that could not be proved
  |
8 | 	ensures alt_sum == sum_arrray(arr, 1)
  |                  ^^

problem_data/p6_6_alternating_sum/program.dfy(14,22): Error: this invariant could not be proved to be maintained by the loop
 Related message: loop invariant violation
   |
14 |     invariant alt_sum == sum_arrray(arr[..i], 1)
   |                       ^^


Dafny program verifier finished with 1 verified, 2 errors

Correct the error by altering only the method body. Produce and show only the Dafny body, including the curly braces that surround it. Do not show the signature nor contract. You must not use 'assume' anywhere in your code.

\end{lstlisting}
\subsection*{GenAI $\rightarrow$ Program}
\textbf{System fingerprint: }
\\\textbf{ID: }chatcmpl-Co5BLAKzjpu0uZdqE8P0DM1GWlWLN
\\\textbf{Model: }gpt-5.1-chat-latest
\\\textbf{Created at: }1766052355
\\\textbf{Finish reason: }stop
\\\textbf{Usage: }2013 tokens in, and 200 tokens out
\begin{lstlisting}
{
  alt_sum := 0;
  var i := 0;
  var sign := 1;
  while i < |arr|
    invariant 0 <= i <= |arr|
    invariant alt_sum == sum_arrray(arr[..i], 1)
    invariant sign == (if i % 2 == 0 then 1 else -1)
  {
    alt_sum := alt_sum + arr[i] * sign;
    i := i + 1;
    sign := -sign;
  }
}
\end{lstlisting}
\subsection*{Program $\rightarrow$ GenAI}
\begin{lstlisting}
When using dafny verify, the below error is emitted and verify fails:

problem_data/p6_6_alternating_sum/program.dfy(9,0): Error: a postcondition could not be proved on this return path
  |
9 | {
  | ^

problem_data/p6_6_alternating_sum/program.dfy(8,17): Related location: this is the postcondition that could not be proved
  |
8 | 	ensures alt_sum == sum_arrray(arr, 1)
  |                  ^^

problem_data/p6_6_alternating_sum/program.dfy(15,22): Error: this invariant could not be proved to be maintained by the loop
 Related message: loop invariant violation
   |
15 |     invariant alt_sum == sum_arrray(arr[..i], 1)
   |                       ^^


Dafny program verifier finished with 1 verified, 2 errors

Correct the error by altering only the method body. Produce and show only the Dafny body, including the curly braces that surround it. Do not show the signature nor contract. You must not use 'assume' anywhere in your code.

\end{lstlisting}
\subsection*{GenAI $\rightarrow$ Program}
\textbf{System fingerprint: }
\\\textbf{ID: }chatcmpl-Co5BQNpgiQsg8qD9ADeYBP7POpoOZ
\\\textbf{Model: }gpt-5.1-chat-latest
\\\textbf{Created at: }1766052360
\\\textbf{Finish reason: }stop
\\\textbf{Usage: }2388 tokens in, and 525 tokens out
\begin{lstlisting}
{
  alt_sum := 0;
  var i := 0;
  while i < |arr|
    invariant 0 <= i <= |arr|
    invariant alt_sum + sum_arrray(arr[i..], if i % 2 == 0 then 1 else -1) == sum_arrray(arr, 1)
  {
    var sign := if i % 2 == 0 then 1 else -1;
    alt_sum := alt_sum + arr[i] * sign;
    i := i + 1;
  }
}
\end{lstlisting}
\section*{Final Program}
\begin{lstlisting}
function sum_arrray(arr : seq<int>, i : int) : int
{
  if |arr| == 0 then 0
  else arr[0] * i + sum_arrray(arr[1..], i * -1)
}

method p6_6_alternating_sum(arr: seq<int>) returns (alt_sum: int)
	ensures alt_sum == sum_arrray(arr, 1)
{
  alt_sum := 0;
  var i := 0;
  while i < |arr|
    invariant 0 <= i <= |arr|
    invariant alt_sum + sum_arrray(arr[i..], if i % 2 == 0 then 1 else -1) == sum_arrray(arr, 1)
  {
    var sign := if i % 2 == 0 then 1 else -1;
    alt_sum := alt_sum + arr[i] * sign;
    i := i + 1;
  }
}
\end{lstlisting}
\section*{Total Token Usage}
\textbf{Input tokens: }8798
\\\textbf{Output tokens: }1892
\\\textbf{Reasoning tokens: }1088
\\\textbf{Sum of `total tokens': }10690
\section*{Experiment Timings}
\textbf{Overall Experiment} started at 1766052334436, ended at 1766052371157, lasting 36721ms (36.72 seconds)
\\\textbf{Iteration \#4} started at 1766052352427, ended at 1766052357458, lasting 5031ms (5.03 seconds)
\\\textbf{Iteration \#5} started at 1766052357458, ended at 1766052361800, lasting 4342ms (4.34 seconds)
\\\textbf{Iteration \#6} started at 1766052361800, ended at 1766052371157, lasting 9357ms (9.36 seconds)
\\\textbf{Iteration \#1} started at 1766052334437, ended at 1766052342244, lasting 7807ms (7.81 seconds)
\\\textbf{Iteration \#2} started at 1766052342244, ended at 1766052345558, lasting 3314ms (3.31 seconds)
\\\textbf{Iteration \#3} started at 1766052345558, ended at 1766052352427, lasting 6869ms (6.87 seconds)
\end{document}
