\documentclass{article}
\usepackage{lmodern}
\usepackage{amsmath}
\usepackage{listings}
\usepackage{fullpage}
\usepackage{parskip}
\usepackage{xcolor}
\lstset{
	basicstyle=\ttfamily,
	columns=fullflexible,
	frame=single,
	breaklines=true,
	postbreak=\mbox{\textcolor{red}{$\hookrightarrow$}\space},
}
\begin{document}
\title{Experiment `p6\_2\_b\_shift\_right' Results}
\author{\today}
\date{}
\maketitle
\textbf{Experiment outcome: }SUCCESS
\\\textbf{Bad responses: }0
\\\textbf{Responses containing}~\texttt{assume}~\textbf{: }0
\\\textbf{Resolution attempts: }1
\\\textbf{Hard fails (resolution): }0
\\\textbf{Soft fails (resolution): }0
\\\textbf{Verification attempts: }1
\section*{Problem Specification}
\textbf{Problem name: }p6\_2\_b\_shift\_right
\\\textbf{Natural language statement: }Write a method to shift all elements of an array by one to the right and move the last element into the first position. For example, [1, 4, 9, 16, 25] would be transformed into [25, 1, 4, 9, 16].
\\\textbf{Method signature: }p6\_2\_b\_shift\_right(arr: seq<int>) returns (result: seq<int>)
\subsection*{Ensures}
\begin{itemize}
\item \texttt{|result| == |arr|}
\item \texttt{result[0] == arr[|arr| - 1]}
\item \texttt{forall i :: 1 <= i < |arr| ==> result[i] == arr[i - 1]}
\end{itemize}
\subsection*{Requires}
\begin{itemize}
\item \texttt{|arr| >= 1}
\end{itemize}
\clearpage
\section*{GenAI interactions}
Below you will find all interactions between the `user' (program) and the `assistant' (OpenAI).
\subsection*{Program $\rightarrow$ GenAI}
\begin{lstlisting}
You are given the following task to perform in Dafny:

Write a method to shift all elements of an array by one to the right and move the last element into the first position. For example, [1, 4, 9, 16, 25] would be transformed into [25, 1, 4, 9, 16].

The signature should be:

method p6_2_b_shift_right(arr: seq<int>) returns (result: seq<int>)

The method should respect the following contract:

requires |arr| >= 1, ensures |result| == |arr|, ensures result[0] == arr[|arr| - 1], ensures forall i :: 1 <= i < |arr| ==> result[i] == arr[i - 1]

Produce and show only the Dafny body of this method, including the curly braces that surround it. Do not show the signature nor contract. You must not use 'assume' anywhere in your code.

\end{lstlisting}
\subsection*{GenAI $\rightarrow$ Program}
\textbf{System fingerprint: }
\\\textbf{ID: }chatcmpl-Co55v8juI9vcrfoEhcFQ1AutfF23i
\\\textbf{Model: }gpt-5.1-chat-latest
\\\textbf{Created at: }1766052019
\\\textbf{Finish reason: }stop
\\\textbf{Usage: }444 tokens in, and 116 tokens out
\begin{lstlisting}
{
  var n := |arr|;
  result := arr[n - 1 .. n] + arr[0 .. n - 1];
}
\end{lstlisting}
\section*{Final Program}
\begin{lstlisting}
method p6_2_b_shift_right(arr: seq<int>) returns (result: seq<int>)
	requires |arr| >= 1
	ensures |result| == |arr|
	ensures result[0] == arr[|arr| - 1]
	ensures forall i :: 1 <= i < |arr| ==> result[i] == arr[i - 1]
{
  var n := |arr|;
  result := arr[n - 1 .. n] + arr[0 .. n - 1];
}
\end{lstlisting}
\section*{Total Token Usage}
\textbf{Input tokens: }444
\\\textbf{Output tokens: }116
\\\textbf{Reasoning tokens: }64
\\\textbf{Sum of `total tokens': }560
\section*{Experiment Timings}
\textbf{Overall Experiment} started at 1766052020688, ended at 1766052024297, lasting 3609ms (3.61 seconds)
\\\textbf{Iteration \#1} started at 1766052020688, ended at 1766052024297, lasting 3609ms (3.61 seconds)
\end{document}
