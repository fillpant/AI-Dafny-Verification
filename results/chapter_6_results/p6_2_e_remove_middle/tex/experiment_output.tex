\documentclass{article}
\usepackage{lmodern}
\usepackage{amsmath}
\usepackage{listings}
\usepackage{fullpage}
\usepackage{parskip}
\usepackage{xcolor}
\lstset{
	basicstyle=\ttfamily,
	columns=fullflexible,
	frame=single,
	breaklines=true,
	postbreak=\mbox{\textcolor{red}{$\hookrightarrow$}\space},
}
\begin{document}
\title{Experiment `p6\_2\_e\_remove\_middle' Results}
\author{\today}
\date{}
\maketitle
\textbf{Experiment outcome: }SUCCESS
\\\textbf{Bad responses: }0
\\\textbf{Responses containing}~\texttt{assume}~\textbf{: }0
\\\textbf{Resolution attempts: }1
\\\textbf{Hard fails (resolution): }0
\\\textbf{Soft fails (resolution): }0
\\\textbf{Verification attempts: }1
\section*{Problem Specification}
\textbf{Problem name: }p6\_2\_e\_remove\_middle
\\\textbf{Natural language statement: }Write a method to remove the middle element of an array if the array length is odd, or the middle two elements if the length is even.
\\\textbf{Method signature: }p6\_2\_e\_remove\_middle(arr: seq<int>) returns (result: seq<int>)
\subsection*{Ensures}
\begin{itemize}
\item \texttt{|result| == if |arr| \% 2 == 1 then |arr| - 1 else |arr| - 2}
\item \texttt{|arr| \% 2 == 1 ==> forall i :: 0 <= i < |arr| / 2 ==> result[i] == arr[i] \&\& forall j :: |arr| / 2 <= j < |result| ==> result[j] == arr[j + 1]}
\item \texttt{|arr| \% 2 == 0 ==> forall i :: 0 <= i < |arr| / 2 - 1 ==> result[i] == arr[i] \&\& forall j :: |arr| / 2 - 1 <= j < |result| ==> result[j] == arr[j + 2]}
\end{itemize}
\subsection*{Requires}
\begin{itemize}
\item \texttt{|arr| >= 1}
\end{itemize}
\clearpage
\section*{GenAI interactions}
Below you will find all interactions between the `user' (program) and the `assistant' (OpenAI).
\subsection*{Program $\rightarrow$ GenAI}
\begin{lstlisting}
You are given the following task to perform in Dafny:

Write a method to remove the middle element of an array if the array length is odd, or the middle two elements if the length is even.

The signature should be:

method p6_2_e_remove_middle(arr: seq<int>) returns (result: seq<int>)

The method should respect the following contract:

requires |arr| >= 1, ensures |result| == if |arr| % 2 == 1 then |arr| - 1 else |arr| - 2, ensures |arr| % 2 == 1 ==> forall i :: 0 <= i < |arr| / 2 ==> result[i] == arr[i] && forall j :: |arr| / 2 <= j < |result| ==> result[j] == arr[j + 1], ensures |arr| % 2 == 0 ==> forall i :: 0 <= i < |arr| / 2 - 1 ==> result[i] == arr[i] && forall j :: |arr| / 2 - 1 <= j < |result| ==> result[j] == arr[j + 2]

Produce and show only the Dafny body of this method, including the curly braces that surround it. Do not show the signature nor contract. You must not use 'assume' anywhere in your code.

\end{lstlisting}
\subsection*{GenAI $\rightarrow$ Program}
\textbf{System fingerprint: }
\\\textbf{ID: }chatcmpl-Co56nbIK7sXqc83VFYUsdTRAIDdTG
\\\textbf{Model: }gpt-5.1-chat-latest
\\\textbf{Created at: }1766052073
\\\textbf{Finish reason: }stop
\\\textbf{Usage: }596 tokens in, and 180 tokens out
\begin{lstlisting}
{
  if |arr| % 2 == 1 {
    var mid := |arr| / 2;
    result := arr[..mid] + arr[mid+1..];
  } else {
    var mid1 := |arr| / 2 - 1;
    var mid2 := |arr| / 2;
    result := arr[..mid1] + arr[mid2+1..];
  }
}
\end{lstlisting}
\section*{Final Program}
\begin{lstlisting}
method p6_2_e_remove_middle(arr: seq<int>) returns (result: seq<int>)
	requires |arr| >= 1
	ensures |result| == if |arr| % 2 == 1 then |arr| - 1 else |arr| - 2
	ensures |arr| % 2 == 1 ==> forall i :: 0 <= i < |arr| / 2 ==> result[i] == arr[i] && forall j :: |arr| / 2 <= j < |result| ==> result[j] == arr[j + 1]
	ensures |arr| % 2 == 0 ==> forall i :: 0 <= i < |arr| / 2 - 1 ==> result[i] == arr[i] && forall j :: |arr| / 2 - 1 <= j < |result| ==> result[j] == arr[j + 2]
{
  if |arr| % 2 == 1 {
    var mid := |arr| / 2;
    result := arr[..mid] + arr[mid+1..];
  } else {
    var mid1 := |arr| / 2 - 1;
    var mid2 := |arr| / 2;
    result := arr[..mid1] + arr[mid2+1..];
  }
}
\end{lstlisting}
\section*{Total Token Usage}
\textbf{Input tokens: }596
\\\textbf{Output tokens: }180
\\\textbf{Reasoning tokens: }64
\\\textbf{Sum of `total tokens': }776
\section*{Experiment Timings}
\textbf{Overall Experiment} started at 1766052075296, ended at 1766052079355, lasting 4059ms (4.06 seconds)
\\\textbf{Iteration \#1} started at 1766052075296, ended at 1766052079355, lasting 4059ms (4.06 seconds)
\end{document}
