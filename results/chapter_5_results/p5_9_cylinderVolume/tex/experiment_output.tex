\documentclass{article}
\usepackage{lmodern}
\usepackage{amsmath}
\usepackage{listings}
\usepackage{fullpage}
\usepackage{parskip}
\usepackage{xcolor}
\lstset{
	basicstyle=\ttfamily,
	columns=fullflexible,
	frame=single,
	breaklines=true,
	postbreak=\mbox{\textcolor{red}{$\hookrightarrow$}\space},
}
\begin{document}
\title{Experiment `p5\_9\_cylinderVolume' Results}
\author{\today}
\date{}
\maketitle
\textbf{Experiment outcome: }SUCCESS
\\\textbf{Bad responses: }0
\\\textbf{Responses containing}~\texttt{assume}~\textbf{: }0
\\\textbf{Resolution attempts: }1
\\\textbf{Hard fails (resolution): }0
\\\textbf{Soft fails (resolution): }0
\\\textbf{Verification attempts: }1
\section*{Problem Specification}
\textbf{Problem name: }p5\_9\_cylinderVolume
\\\textbf{Natural language statement: }Write a method that computes the volume of a cylinder with a circular base, given radius and height.
\\\textbf{Method signature: }p5\_9\_cylinderVolume(r: real, h: real) returns (volume: real)
\subsection*{Ensures}
\begin{itemize}
\item \texttt{volume == 3.14159 * r * r * h}
\end{itemize}
\clearpage
\section*{GenAI interactions}
Below you will find all interactions between the `user' (program) and the `assistant' (OpenAI).
\subsection*{Program $\rightarrow$ GenAI}
\begin{lstlisting}
You are given the following task to perform in Dafny:

Write a method that computes the volume of a cylinder with a circular base, given radius and height.

The signature should be:

method p5_9_cylinderVolume(r: real, h: real) returns (volume: real)

The method should respect the following contract:

ensures volume == 3.14159 * r * r * h

Produce and show only the Dafny body of this method, including the curly braces that surround it. Do not show the signature nor contract. You must not use 'assume' anywhere in your code.

\end{lstlisting}
\subsection*{GenAI $\rightarrow$ Program}
\textbf{System fingerprint: }
\\\textbf{ID: }chatcmpl-Co4xygHoySaHtQiZxd0ZKVeefmVvP
\\\textbf{Model: }gpt-5.1-chat-latest
\\\textbf{Created at: }1766051526
\\\textbf{Finish reason: }stop
\\\textbf{Usage: }284 tokens in, and 100 tokens out
\begin{lstlisting}
{
  volume := 3.14159 * r * r * h;
}
\end{lstlisting}
\section*{Final Program}
\begin{lstlisting}
method p5_9_cylinderVolume(r: real, h: real) returns (volume: real)
	ensures volume == 3.14159 * r * r * h
{
  volume := 3.14159 * r * r * h;
}
\end{lstlisting}
\section*{Total Token Usage}
\textbf{Input tokens: }284
\\\textbf{Output tokens: }100
\\\textbf{Reasoning tokens: }64
\\\textbf{Sum of `total tokens': }384
\section*{Experiment Timings}
\textbf{Overall Experiment} started at 1766051528530, ended at 1766051531660, lasting 3130ms (3.13 seconds)
\\\textbf{Iteration \#1} started at 1766051528530, ended at 1766051531660, lasting 3130ms (3.13 seconds)
\end{document}
