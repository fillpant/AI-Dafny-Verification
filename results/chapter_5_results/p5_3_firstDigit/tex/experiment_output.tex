\documentclass{article}
\usepackage{lmodern}
\usepackage{amsmath}
\usepackage{listings}
\usepackage{fullpage}
\usepackage{parskip}
\usepackage{xcolor}
\lstset{
	basicstyle=\ttfamily,
	columns=fullflexible,
	frame=single,
	breaklines=true,
	postbreak=\mbox{\textcolor{red}{$\hookrightarrow$}\space},
}
\begin{document}
\title{Experiment `p5\_3\_firstDigit' Results}
\author{\today}
\date{}
\maketitle
\textbf{Experiment outcome: }SUCCESS
\\\textbf{Bad responses: }0
\\\textbf{Responses containing}~\texttt{assume}~\textbf{: }0
\\\textbf{Resolution attempts: }1
\\\textbf{Hard fails (resolution): }0
\\\textbf{Soft fails (resolution): }0
\\\textbf{Verification attempts: }1
\section*{Problem Specification}
\textbf{Problem name: }p5\_3\_firstDigit
\\\textbf{Natural language statement: }Write a method returning the first digit of the argument.
\\\textbf{Method signature: }p5\_3\_firstDigit(n: int) returns (first: int)
\subsection*{Ensures}
\begin{itemize}
\item \texttt{0 <= first < 10}
\item \texttt{intToSeq(n)[0] == first}
\end{itemize}
\subsection*{Requires}
\begin{itemize}
\item \texttt{0 <= n}
\end{itemize}
\subsection*{Functional Code Given}
\begin{lstlisting}
function intToSeq(n: int) : seq<int>
  requires n >= 0
  ensures forall d :: d in intToSeq(n) ==> 0 <= d < 10
{
  if n < 10 then [n]
  else intToSeq(n / 10) + [n % 10]
}
\end{lstlisting}
\clearpage
\section*{GenAI interactions}
Below you will find all interactions between the `user' (program) and the `assistant' (OpenAI).
\subsection*{Program $\rightarrow$ GenAI}
\begin{lstlisting}
You are given the following task to perform in Dafny:

Write a method returning the first digit of the argument.

The signature should be:

method p5_3_firstDigit(n: int) returns (first: int)

The method should respect the following contract:

requires 0 <= n, ensures 0 <= first < 10, ensures intToSeq(n)[0] == first

The contract uses the following dafny code:

function intToSeq(n: int) : seq<int>
  requires n >= 0
  ensures forall d :: d in intToSeq(n) ==> 0 <= d < 10
{
  if n < 10 then [n]
  else intToSeq(n / 10) + [n % 10]
}

These function(s) must not be used in your implementation of the method.

Produce and show only the Dafny body of this method, including the curly braces that surround it. Do not show the signature nor contract. You must not use 'assume' anywhere in your code.

\end{lstlisting}
\subsection*{GenAI $\rightarrow$ Program}
\textbf{System fingerprint: }
\\\textbf{ID: }chatcmpl-Co4tg8JuTVT15L7Pp977x76N6VWKY
\\\textbf{Model: }gpt-5.1-chat-latest
\\\textbf{Created at: }1766051260
\\\textbf{Finish reason: }stop
\\\textbf{Usage: }462 tokens in, and 313 tokens out
\begin{lstlisting}
{
  if n < 10 {
    first := n;
  } else {
    first := p5_3_firstDigit(n / 10);
  }
}
\end{lstlisting}
\section*{Final Program}
\begin{lstlisting}
function intToSeq(n: int) : seq<int>
  requires n >= 0
  ensures forall d :: d in intToSeq(n) ==> 0 <= d < 10
{
  if n < 10 then [n]
  else intToSeq(n / 10) + [n % 10]
}

method p5_3_firstDigit(n: int) returns (first: int)
	requires 0 <= n
	ensures 0 <= first < 10
	ensures intToSeq(n)[0] == first
{
  if n < 10 {
    first := n;
  } else {
    first := p5_3_firstDigit(n / 10);
  }
}
\end{lstlisting}
\section*{Total Token Usage}
\textbf{Input tokens: }462
\\\textbf{Output tokens: }313
\\\textbf{Reasoning tokens: }256
\\\textbf{Sum of `total tokens': }775
\section*{Experiment Timings}
\textbf{Overall Experiment} started at 1766051261986, ended at 1766051267884, lasting 5898ms (5.90 seconds)
\\\textbf{Iteration \#1} started at 1766051261987, ended at 1766051267884, lasting 5897ms (5.90 seconds)
\end{document}
