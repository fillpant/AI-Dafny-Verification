\documentclass{article}
\usepackage{lmodern}
\usepackage{amsmath}
\usepackage{listings}
\usepackage{fullpage}
\usepackage{parskip}
\usepackage{xcolor}
\lstset{
	basicstyle=\ttfamily,
	columns=fullflexible,
	frame=single,
	breaklines=true,
	postbreak=\mbox{\textcolor{red}{$\hookrightarrow$}\space},
}
\begin{document}
\title{Experiment `p\_1\_2' Results}
\author{\today}
\date{}
\maketitle
\textbf{Experiment outcome: }SUCCESS
\\\textbf{Bad responses: }0
\\\textbf{Responses containing}~\texttt{assume}~\textbf{: }0
\\\textbf{Resolution attempts: }1
\\\textbf{Hard fails (resolution): }0
\\\textbf{Soft fails (resolution): }0
\\\textbf{Verification attempts: }1
\section*{Problem Specification}
\textbf{Problem name: }p\_1\_2
\\\textbf{Natural language statement: }Write a method that returns the sum of the first ten positive integers, 1 + 2 + ... + 10.
\\\textbf{Method signature: }p\_1\_2() returns (i : int)
\subsection*{Ensures}
\begin{itemize}
\item \texttt{i == (10 * 11) / 2}
\end{itemize}
\clearpage
\section*{GenAI interactions}
Below you will find all interactions between the `user' (program) and the `assistant' (OpenAI).
\subsection*{Program $\rightarrow$ GenAI}
\begin{lstlisting}
You are given the following task to perform in Dafny:

Write a method that returns the sum of the first ten positive integers, 1 + 2 + ... + 10.

The signature should be:

method p_1_2() returns (i : int)

The method should respect the following contract:

ensures i == (10 * 11) / 2

Produce and show only the Dafny body of this method, including the curly braces that surround it. Do not show the signature nor contract. You must not use 'assume' anywhere in your code.

\end{lstlisting}
\subsection*{GenAI $\rightarrow$ Program}
\textbf{System fingerprint: }
\\\textbf{ID: }chatcmpl-Co4LhWDjlOmq3NW72OJRVh9RQMxdj
\\\textbf{Model: }gpt-5.1-chat-latest
\\\textbf{Created at: }1766049153
\\\textbf{Finish reason: }stop
\\\textbf{Usage: }272 tokens in, and 228 tokens out
\begin{lstlisting}
{
  var sum := 0;
  var k := 1;
  while k <= 10
    invariant 1 <= k && k <= 11
    invariant sum == (k - 1) * k / 2
  {
    sum := sum + k;
    k := k + 1;
  }
  i := sum;
}
\end{lstlisting}
\section*{Final Program}
\begin{lstlisting}
method p_1_2() returns (i : int)
	ensures i == (10 * 11) / 2
{
  var sum := 0;
  var k := 1;
  while k <= 10
    invariant 1 <= k && k <= 11
    invariant sum == (k - 1) * k / 2
  {
    sum := sum + k;
    k := k + 1;
  }
  i := sum;
}
\end{lstlisting}
\section*{Total Token Usage}
\textbf{Input tokens: }272
\\\textbf{Output tokens: }228
\\\textbf{Reasoning tokens: }128
\\\textbf{Sum of `total tokens': }500
\section*{Experiment Timings}
\textbf{Overall Experiment} started at 1766049154896, ended at 1766049158737, lasting 3841ms (3.84 seconds)
\\\textbf{Iteration \#1} started at 1766049154897, ended at 1766049158737, lasting 3840ms (3.84 seconds)
\end{document}
