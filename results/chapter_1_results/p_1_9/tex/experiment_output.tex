\documentclass{article}
\usepackage{lmodern}
\usepackage{amsmath}
\usepackage{listings}
\usepackage{fullpage}
\usepackage{parskip}
\usepackage{xcolor}
\lstset{
	basicstyle=\ttfamily,
	columns=fullflexible,
	frame=single,
	breaklines=true,
	postbreak=\mbox{\textcolor{red}{$\hookrightarrow$}\space},
}
\begin{document}
\title{Experiment `p\_1\_9' Results}
\author{\today}
\date{}
\maketitle
\textbf{Experiment outcome: }SUCCESS
\\\textbf{Bad responses: }0
\\\textbf{Responses containing}~\texttt{assume}~\textbf{: }0
\\\textbf{Resolution attempts: }1
\\\textbf{Hard fails (resolution): }0
\\\textbf{Soft fails (resolution): }0
\\\textbf{Verification attempts: }1
\section*{Problem Specification}
\textbf{Problem name: }p\_1\_9
\\\textbf{Natural language statement: }Write a method that returns a house in ASCII art.
\\\textbf{Method signature: }p\_1\_9() returns (s : string)
\subsection*{Ensures}
\begin{itemize}
\item \texttt{s == "   +\textbackslash{}n  + +\textbackslash{}n +   +\textbackslash{}n+-----+\textbackslash{}n| .-. |\textbackslash{}n| | | |\textbackslash{}n+-+-+-+\textbackslash{}n"}
\end{itemize}
\clearpage
\section*{GenAI interactions}
Below you will find all interactions between the `user' (program) and the `assistant' (OpenAI).
\subsection*{Program $\rightarrow$ GenAI}
\begin{lstlisting}
You are given the following task to perform in Dafny:

Write a method that returns a house in ASCII art.

The signature should be:

method p_1_9() returns (s : string)

The method should respect the following contract:

ensures s == "   +\n  + +\n +   +\n+-----+\n| .-. |\n| | | |\n+-+-+-+\n"

Produce and show only the Dafny body of this method, including the curly braces that surround it. Do not show the signature nor contract. You must not use 'assume' anywhere in your code.

\end{lstlisting}
\subsection*{GenAI $\rightarrow$ Program}
\textbf{System fingerprint: }
\\\textbf{ID: }chatcmpl-Co4MGJ6jMxX75fbOl9E9KKHurpE2L
\\\textbf{Model: }gpt-5.1-chat-latest
\\\textbf{Created at: }1766049188
\\\textbf{Finish reason: }stop
\\\textbf{Usage: }292 tokens in, and 127 tokens out
\begin{lstlisting}
{
  s := "   +\n  + +\n +   +\n+-----+\n| .-. |\n| | | |\n+-+-+-+\n";
}
\end{lstlisting}
\section*{Final Program}
\begin{lstlisting}
method p_1_9() returns (s : string)
	ensures s == "   +\n  + +\n +   +\n+-----+\n| .-. |\n| | | |\n+-+-+-+\n"
{
  s := "   +\n  + +\n +   +\n+-----+\n| .-. |\n| | | |\n+-+-+-+\n";
}
\end{lstlisting}
\section*{Total Token Usage}
\textbf{Input tokens: }292
\\\textbf{Output tokens: }127
\\\textbf{Reasoning tokens: }64
\\\textbf{Sum of `total tokens': }419
\section*{Experiment Timings}
\textbf{Overall Experiment} started at 1766049190180, ended at 1766049193070, lasting 2890ms (2.89 seconds)
\\\textbf{Iteration \#1} started at 1766049190180, ended at 1766049193070, lasting 2890ms (2.89 seconds)
\end{document}
