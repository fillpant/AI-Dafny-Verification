\documentclass{article}
\usepackage{lmodern}
\usepackage{amsmath}
\usepackage{listings}
\usepackage{fullpage}
\usepackage{parskip}
\usepackage{xcolor}
\lstset{
	basicstyle=\ttfamily,
	columns=fullflexible,
	frame=single,
	breaklines=true,
	postbreak=\mbox{\textcolor{red}{$\hookrightarrow$}\space},
}
\begin{document}
\title{Experiment `p\_1\_3' Results}
\author{\today}
\date{}
\maketitle
\textbf{Experiment outcome: }SUCCESS
\\\textbf{Bad responses: }0
\\\textbf{Responses containing}~\texttt{assume}~\textbf{: }0
\\\textbf{Resolution attempts: }1
\\\textbf{Hard fails (resolution): }0
\\\textbf{Soft fails (resolution): }0
\\\textbf{Verification attempts: }1
\section*{Problem Specification}
\textbf{Problem name: }p\_1\_3
\\\textbf{Natural language statement: }Write a method that returns the product of the first ten positive integers, 1 * 2 * ... * 10.
\\\textbf{Method signature: }p\_1\_3() returns (i : int)
\subsection*{Ensures}
\begin{itemize}
\item \texttt{i == factorial(10)}
\end{itemize}
\subsection*{Functional Code Given}
\begin{lstlisting}
function factorial(n : int) : int
  decreases n
{
  if n <= 1 then 1 else n * factorial(n - 1)
}
\end{lstlisting}
\clearpage
\section*{GenAI interactions}
Below you will find all interactions between the `user' (program) and the `assistant' (OpenAI).
\subsection*{Program $\rightarrow$ GenAI}
\begin{lstlisting}
You are given the following task to perform in Dafny:

Write a method that returns the product of the first ten positive integers, 1 * 2 * ... * 10.

The signature should be:

method p_1_3() returns (i : int)

The method should respect the following contract:

ensures i == factorial(10)

The contract uses the following dafny code:

function factorial(n : int) : int
  decreases n
{
  if n <= 1 then 1 else n * factorial(n - 1)
}

These function(s) must not be used in your implementation of the method.

Produce and show only the Dafny body of this method, including the curly braces that surround it. Do not show the signature nor contract. You must not use 'assume' anywhere in your code.

\end{lstlisting}
\subsection*{GenAI $\rightarrow$ Program}
\textbf{System fingerprint: }
\\\textbf{ID: }chatcmpl-Co4Ll7tntL5DAeCiwgpKSfVgPFZtY
\\\textbf{Model: }gpt-5.1-chat-latest
\\\textbf{Created at: }1766049157
\\\textbf{Finish reason: }stop
\\\textbf{Usage: }374 tokens in, and 374 tokens out
\begin{lstlisting}
{
  i := 1 * 2 * 3 * 4 * 5 * 6 * 7 * 8 * 9 * 10;
}
\end{lstlisting}
\section*{Final Program}
\begin{lstlisting}
function factorial(n : int) : int
  decreases n
{
  if n <= 1 then 1 else n * factorial(n - 1)
}

method p_1_3() returns (i : int)
	ensures i == factorial(10)
{
  i := 1 * 2 * 3 * 4 * 5 * 6 * 7 * 8 * 9 * 10;
}
\end{lstlisting}
\section*{Total Token Usage}
\textbf{Input tokens: }374
\\\textbf{Output tokens: }374
\\\textbf{Reasoning tokens: }320
\\\textbf{Sum of `total tokens': }748
\section*{Experiment Timings}
\textbf{Overall Experiment} started at 1766049159361, ended at 1766049165251, lasting 5890ms (5.89 seconds)
\\\textbf{Iteration \#1} started at 1766049159362, ended at 1766049165251, lasting 5889ms (5.89 seconds)
\end{document}
