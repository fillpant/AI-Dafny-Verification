\documentclass{article}
\usepackage{lmodern}
\usepackage{amsmath}
\usepackage{listings}
\usepackage{fullpage}
\usepackage{parskip}
\usepackage{xcolor}
\lstset{
	basicstyle=\ttfamily,
	columns=fullflexible,
	frame=single,
	breaklines=true,
	postbreak=\mbox{\textcolor{red}{$\hookrightarrow$}\space},
}
\begin{document}
\title{Experiment `p4\_6\_find\_minimum' Results}
\author{\today}
\date{}
\maketitle
\textbf{Experiment outcome: }SUCCESS
\\\textbf{Bad responses: }0
\\\textbf{Responses containing}~\texttt{assume}~\textbf{: }0
\\\textbf{Resolution attempts: }3
\\\textbf{Hard fails (resolution): }0
\\\textbf{Soft fails (resolution): }1
\\\textbf{Verification attempts: }2
\section*{Problem Specification}
\textbf{Problem name: }p4\_6\_find\_minimum
\\\textbf{Natural language statement: }Write a method that finds the minimum value from a list of inputs into a method.
\\\textbf{Method signature: }p4\_6\_find\_minimum(inputs: array<int>) returns (min: int)
\subsection*{Ensures}
\begin{itemize}
\item \texttt{forall i :: 0 <= i < inputs.Length ==> min <= inputs[i]}
\item \texttt{exists i :: 0 <= i < inputs.Length \&\& min == inputs[i]}
\end{itemize}
\subsection*{Requires}
\begin{itemize}
\item \texttt{0 < inputs.Length}
\end{itemize}
\clearpage
\section*{GenAI interactions}
Below you will find all interactions between the `user' (program) and the `assistant' (OpenAI).
\subsection*{Program $\rightarrow$ GenAI}
\begin{lstlisting}
You are given the following task to perform in Dafny:

Write a method that finds the minimum value from a list of inputs into a method.

The signature should be:

method p4_6_find_minimum(inputs: array<int>) returns (min: int)

The method should respect the following contract:

requires 0 < inputs.Length, ensures forall i :: 0 <= i < inputs.Length ==> min <= inputs[i], ensures exists i :: 0 <= i < inputs.Length && min == inputs[i]

Produce and show only the Dafny body of this method, including the curly braces that surround it. Do not show the signature nor contract. You must not use 'assume' anywhere in your code.

\end{lstlisting}
\subsection*{GenAI $\rightarrow$ Program}
\textbf{System fingerprint: }
\\\textbf{ID: }chatcmpl-Co4ejFoVBGtDTV1FajCrsDenhnaSW
\\\textbf{Model: }gpt-5.1-chat-latest
\\\textbf{Created at: }1766050333
\\\textbf{Finish reason: }stop
\\\textbf{Usage: }322 tokens in, and 197 tokens out
\begin{lstlisting}
{
  var i := 0;
  min := inputs[0];
  // Loop to scan all elements
  while i < inputs.Length
    invariant 0 <= i <= inputs.Length
    invariant forall j :: 0 <= j < i ==> min <= inputs[j]
    invariant exists j :: 0 <= j < i ==> min == inputs[j]
  {
    if inputs[i] < min {
      min := inputs[i];
    }
    i := i + 1;
  }
}
\end{lstlisting}
\subsection*{Program $\rightarrow$ GenAI}
\begin{lstlisting}
When using dafny resolve, the below error is emitted and resolve fails:

problem_data/p4_6_find_minimum/program.dfy(12,14): Warning: the quantifier has the form 'exists x :: A ==> B', which most often is a typo for 'exists x :: A && B'; if you think otherwise, rewrite as 'exists x :: (A ==> B)' or 'exists x :: !A || B' to suppress this warning
   |
12 |     invariant exists j :: 0 <= j < i ==> min == inputs[j]
   |               ^^^^^^

Compilation failed because warnings were found and --allow-warnings is false

Correct the error by altering only the method body. Produce and show only the Dafny body, including the curly braces that surround it. Do not show the signature nor contract. You must not use 'assume' anywhere in your code.

\end{lstlisting}
\subsection*{GenAI $\rightarrow$ Program}
\textbf{System fingerprint: }
\\\textbf{ID: }chatcmpl-Co4en7gJ7UOQjNdudt5JRfJ2tmQ9u
\\\textbf{Model: }gpt-5.1-chat-latest
\\\textbf{Created at: }1766050337
\\\textbf{Finish reason: }stop
\\\textbf{Usage: }679 tokens in, and 132 tokens out
\begin{lstlisting}
{
  var i := 0;
  min := inputs[0];
  while i < inputs.Length
    invariant 0 <= i <= inputs.Length
    invariant forall j :: 0 <= j < i ==> min <= inputs[j]
    invariant i > 0 ==> (exists j :: 0 <= j < i && min == inputs[j])
  {
    if inputs[i] < min {
      min := inputs[i];
    }
    i := i + 1;
  }
}
\end{lstlisting}
\subsection*{Program $\rightarrow$ GenAI}
\begin{lstlisting}
When using dafny verify, the below error is emitted and verify fails:

problem_data/p4_6_find_minimum/program.dfy(11,25): Error: this invariant could not be proved to be maintained by the loop
 Related message: loop invariant violation
   |
11 |     invariant i > 0 ==> (exists j :: 0 <= j < i && min == inputs[j])
   |                          ^^^^^^


Dafny program verifier finished with 1 verified, 1 error

Correct the error by altering only the method body. Produce and show only the Dafny body, including the curly braces that surround it. Do not show the signature nor contract. You must not use 'assume' anywhere in your code.

\end{lstlisting}
\subsection*{GenAI $\rightarrow$ Program}
\textbf{System fingerprint: }
\\\textbf{ID: }chatcmpl-Co4erR7qkPrJHIRJYbrmU945N3uyb
\\\textbf{Model: }gpt-5.1-chat-latest
\\\textbf{Created at: }1766050341
\\\textbf{Finish reason: }stop
\\\textbf{Usage: }931 tokens in, and 206 tokens out
\begin{lstlisting}
{
  var minIndex := 0;
  min := inputs[0];
  var i := 1;
  while i < inputs.Length
    invariant 1 <= i <= inputs.Length
    invariant 0 <= minIndex < i
    invariant min == inputs[minIndex]
    invariant forall j :: 0 <= j < i ==> min <= inputs[j]
  {
    if inputs[i] < min {
      min := inputs[i];
      minIndex := i;
    }
    i := i + 1;
  }
}
\end{lstlisting}
\section*{Final Program}
\begin{lstlisting}
method p4_6_find_minimum(inputs: array<int>) returns (min: int)
	requires 0 < inputs.Length
	ensures forall i :: 0 <= i < inputs.Length ==> min <= inputs[i]
	ensures exists i :: 0 <= i < inputs.Length && min == inputs[i]
{
  var minIndex := 0;
  min := inputs[0];
  var i := 1;
  while i < inputs.Length
    invariant 1 <= i <= inputs.Length
    invariant 0 <= minIndex < i
    invariant min == inputs[minIndex]
    invariant forall j :: 0 <= j < i ==> min <= inputs[j]
  {
    if inputs[i] < min {
      min := inputs[i];
      minIndex := i;
    }
    i := i + 1;
  }
}
\end{lstlisting}
\section*{Total Token Usage}
\textbf{Input tokens: }1932
\\\textbf{Output tokens: }535
\\\textbf{Reasoning tokens: }128
\\\textbf{Sum of `total tokens': }2467
\section*{Experiment Timings}
\textbf{Overall Experiment} started at 1766050335307, ended at 1766050347630, lasting 12323ms (12.32 seconds)
\\\textbf{Iteration \#1} started at 1766050335309, ended at 1766050339083, lasting 3774ms (3.77 seconds)
\\\textbf{Iteration \#2} started at 1766050339083, ended at 1766050342814, lasting 3731ms (3.73 seconds)
\\\textbf{Iteration \#3} started at 1766050342814, ended at 1766050347630, lasting 4816ms (4.82 seconds)
\end{document}
