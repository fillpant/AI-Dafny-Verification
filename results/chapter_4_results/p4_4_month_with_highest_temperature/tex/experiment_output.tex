\documentclass{article}
\usepackage{lmodern}
\usepackage{amsmath}
\usepackage{listings}
\usepackage{fullpage}
\usepackage{parskip}
\usepackage{xcolor}
\lstset{
	basicstyle=\ttfamily,
	columns=fullflexible,
	frame=single,
	breaklines=true,
	postbreak=\mbox{\textcolor{red}{$\hookrightarrow$}\space},
}
\begin{document}
\title{Experiment `p4\_4\_month\_with\_highest\_temperature' Results}
\author{\today}
\date{}
\maketitle
\textbf{Experiment outcome: }SUCCESS
\\\textbf{Bad responses: }0
\\\textbf{Responses containing}~\texttt{assume}~\textbf{: }0
\\\textbf{Resolution attempts: }1
\\\textbf{Hard fails (resolution): }0
\\\textbf{Soft fails (resolution): }0
\\\textbf{Verification attempts: }1
\section*{Problem Specification}
\textbf{Problem name: }p4\_4\_month\_with\_highest\_temperature
\\\textbf{Natural language statement: }Write a method that reads a sequence of twelve temperature values and returns the number of the month with the highest temperature.
\\\textbf{Method signature: }p4\_5\_month\_with\_highest\_temperature(inputs: seq<real>) returns (hottest: int)
\subsection*{Ensures}
\begin{itemize}
\item \texttt{1 <= hottest <= 12}
\item \texttt{forall i :: 0 <= i <= |inputs|-1 ==> inputs[hottest-1] >= inputs[i]}
\end{itemize}
\subsection*{Requires}
\begin{itemize}
\item \texttt{|inputs| == 12}
\end{itemize}
\clearpage
\section*{GenAI interactions}
Below you will find all interactions between the `user' (program) and the `assistant' (OpenAI).
\subsection*{Program $\rightarrow$ GenAI}
\begin{lstlisting}
You are given the following task to perform in Dafny:

Write a method that reads a sequence of twelve temperature values and returns the number of the month with the highest temperature.

The signature should be:

method p4_5_month_with_highest_temperature(inputs: seq<real>) returns (hottest: int)

The method should respect the following contract:

requires |inputs| == 12, ensures 1 <= hottest <= 12, ensures forall i :: 0 <= i <= |inputs|-1 ==> inputs[hottest-1] >= inputs[i]

Produce and show only the Dafny body of this method, including the curly braces that surround it. Do not show the signature nor contract. You must not use 'assume' anywhere in your code.

\end{lstlisting}
\subsection*{GenAI $\rightarrow$ Program}
\textbf{System fingerprint: }
\\\textbf{ID: }chatcmpl-Co4e2byscvOVSQchlY9CwOLkyJnj9
\\\textbf{Model: }gpt-5.1-chat-latest
\\\textbf{Created at: }1766050290
\\\textbf{Finish reason: }stop
\\\textbf{Usage: }342 tokens in, and 158 tokens out
\begin{lstlisting}
{
  var maxIndex := 0;
  var maxVal := inputs[0];
  var i := 1;
  while i < 12
    invariant 1 <= i <= 12
    invariant 0 <= maxIndex < i
    invariant maxVal == inputs[maxIndex]
    invariant forall j :: 0 <= j < i ==> maxVal >= inputs[j]
  {
    if inputs[i] > maxVal {
      maxVal := inputs[i];
      maxIndex := i;
    }
    i := i + 1;
  }
  hottest := maxIndex + 1;
}
\end{lstlisting}
\section*{Final Program}
\begin{lstlisting}
method p4_5_month_with_highest_temperature(inputs: seq<real>) returns (hottest: int)
	requires |inputs| == 12
	ensures 1 <= hottest <= 12
	ensures forall i :: 0 <= i <= |inputs|-1 ==> inputs[hottest-1] >= inputs[i]
{
  var maxIndex := 0;
  var maxVal := inputs[0];
  var i := 1;
  while i < 12
    invariant 1 <= i <= 12
    invariant 0 <= maxIndex < i
    invariant maxVal == inputs[maxIndex]
    invariant forall j :: 0 <= j < i ==> maxVal >= inputs[j]
  {
    if inputs[i] > maxVal {
      maxVal := inputs[i];
      maxIndex := i;
    }
    i := i + 1;
  }
  hottest := maxIndex + 1;
}
\end{lstlisting}
\section*{Total Token Usage}
\textbf{Input tokens: }342
\\\textbf{Output tokens: }158
\\\textbf{Reasoning tokens: }0
\\\textbf{Sum of `total tokens': }500
\section*{Experiment Timings}
\textbf{Overall Experiment} started at 1766050291926, ended at 1766050295548, lasting 3622ms (3.62 seconds)
\\\textbf{Iteration \#1} started at 1766050291926, ended at 1766050295548, lasting 3622ms (3.62 seconds)
\end{document}
