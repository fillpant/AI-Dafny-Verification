\documentclass{article}
\usepackage{lmodern}
\usepackage{amsmath}
\usepackage{listings}
\usepackage{fullpage}
\usepackage{parskip}
\usepackage{xcolor}
\lstset{
	basicstyle=\ttfamily,
	columns=fullflexible,
	frame=single,
	breaklines=true,
	postbreak=\mbox{\textcolor{red}{$\hookrightarrow$}\space},
}
\begin{document}
\title{Experiment `p4\_2\_a\_min\_max' Results}
\author{\today}
\date{}
\maketitle
\textbf{Experiment outcome: }SUCCESS
\\\textbf{Bad responses: }0
\\\textbf{Responses containing}~\texttt{assume}~\textbf{: }0
\\\textbf{Resolution attempts: }1
\\\textbf{Hard fails (resolution): }0
\\\textbf{Soft fails (resolution): }0
\\\textbf{Verification attempts: }1
\section*{Problem Specification}
\textbf{Problem name: }p4\_2\_a\_min\_max
\\\textbf{Natural language statement: }Write a method that takes a sequence of integers and returns the smallest and largest of the inputs.
\\\textbf{Method signature: }p4\_2\_a\_min\_max(inputs: array<int>) returns (min: int, max: int)
\subsection*{Ensures}
\begin{itemize}
\item \texttt{forall i :: 0 <= i < inputs.Length ==> min <= inputs[i]}
\item \texttt{forall i :: 0 <= i < inputs.Length ==> max >= inputs[i]}
\item \texttt{exists i :: 0 <= i < inputs.Length \&\& min == inputs[i]}
\item \texttt{exists i :: 0 <= i < inputs.Length \&\& max == inputs[i]}
\end{itemize}
\subsection*{Requires}
\begin{itemize}
\item \texttt{0 < inputs.Length}
\end{itemize}
\clearpage
\section*{GenAI interactions}
Below you will find all interactions between the `user' (program) and the `assistant' (OpenAI).
\subsection*{Program $\rightarrow$ GenAI}
\begin{lstlisting}
You are given the following task to perform in Dafny:

Write a method that takes a sequence of integers and returns the smallest and largest of the inputs.

The signature should be:

method p4_2_a_min_max(inputs: array<int>) returns (min: int, max: int)

The method should respect the following contract:

requires 0 < inputs.Length, ensures forall i :: 0 <= i < inputs.Length ==> min <= inputs[i], ensures forall i :: 0 <= i < inputs.Length ==> max >= inputs[i], ensures exists i :: 0 <= i < inputs.Length && min == inputs[i], ensures exists i :: 0 <= i < inputs.Length && max == inputs[i]

Produce and show only the Dafny body of this method, including the curly braces that surround it. Do not show the signature nor contract. You must not use 'assume' anywhere in your code.

\end{lstlisting}
\subsection*{GenAI $\rightarrow$ Program}
\textbf{System fingerprint: }
\\\textbf{ID: }chatcmpl-Co4aQubsfiEUiAfp1GiBDUH6e0E1f
\\\textbf{Model: }gpt-5.1-chat-latest
\\\textbf{Created at: }1766050066
\\\textbf{Finish reason: }stop
\\\textbf{Usage: }402 tokens in, and 197 tokens out
\begin{lstlisting}
{
  var i := 0;
  min := inputs[i];
  max := inputs[i];
  i := i + 1;
  while i < inputs.Length
    invariant 1 <= i <= inputs.Length
    invariant forall j :: 0 <= j < i ==> min <= inputs[j]
    invariant forall j :: 0 <= j < i ==> max >= inputs[j]
    invariant exists j :: 0 <= j < i && min == inputs[j]
    invariant exists j :: 0 <= j < i && max == inputs[j]
  {
    if inputs[i] < min {
      min := inputs[i];
    }
    if inputs[i] > max {
      max := inputs[i];
    }
    i := i + 1;
  }
}
\end{lstlisting}
\section*{Final Program}
\begin{lstlisting}
method p4_2_a_min_max(inputs: array<int>) returns (min: int, max: int)
	requires 0 < inputs.Length
	ensures forall i :: 0 <= i < inputs.Length ==> min <= inputs[i]
	ensures forall i :: 0 <= i < inputs.Length ==> max >= inputs[i]
	ensures exists i :: 0 <= i < inputs.Length && min == inputs[i]
	ensures exists i :: 0 <= i < inputs.Length && max == inputs[i]
{
  var i := 0;
  min := inputs[i];
  max := inputs[i];
  i := i + 1;
  while i < inputs.Length
    invariant 1 <= i <= inputs.Length
    invariant forall j :: 0 <= j < i ==> min <= inputs[j]
    invariant forall j :: 0 <= j < i ==> max >= inputs[j]
    invariant exists j :: 0 <= j < i && min == inputs[j]
    invariant exists j :: 0 <= j < i && max == inputs[j]
  {
    if inputs[i] < min {
      min := inputs[i];
    }
    if inputs[i] > max {
      max := inputs[i];
    }
    i := i + 1;
  }
}
\end{lstlisting}
\section*{Total Token Usage}
\textbf{Input tokens: }402
\\\textbf{Output tokens: }197
\\\textbf{Reasoning tokens: }0
\\\textbf{Sum of `total tokens': }599
\section*{Experiment Timings}
\textbf{Overall Experiment} started at 1766050068068, ended at 1766050073106, lasting 5038ms (5.04 seconds)
\\\textbf{Iteration \#1} started at 1766050068068, ended at 1766050073106, lasting 5038ms (5.04 seconds)
\end{document}
