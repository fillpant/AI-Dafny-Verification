\documentclass{article}
\usepackage{lmodern}
\usepackage{amsmath}
\usepackage{listings}
\usepackage{fullpage}
\usepackage{parskip}
\usepackage{xcolor}
\lstset{
	basicstyle=\ttfamily,
	columns=fullflexible,
	frame=single,
	breaklines=true,
	postbreak=\mbox{\textcolor{red}{$\hookrightarrow$}\space},
}
\begin{document}
\title{Experiment `p4\_3\_a\_extract\_uppercase' Results}
\author{\today}
\date{}
\maketitle
\textbf{Experiment outcome: }SUCCESS
\\\textbf{Bad responses: }0
\\\textbf{Responses containing}~\texttt{assume}~\textbf{: }0
\\\textbf{Resolution attempts: }2
\\\textbf{Hard fails (resolution): }1
\\\textbf{Soft fails (resolution): }0
\\\textbf{Verification attempts: }1
\section*{Problem Specification}
\textbf{Problem name: }p4\_3\_a\_extract\_uppercase
\\\textbf{Natural language statement: }Write a method that takes a string and returns only the uppercase letters in the string.
\\\textbf{Method signature: }p4\_3\_a\_extract\_uppercase(s: string) returns (uppercase: string)
\subsection*{Ensures}
\begin{itemize}
\item \texttt{forall c :: c in uppercase ==> 'A' <= c <= 'Z'}
\item \texttt{forall c :: c in s \&\& 'A' <= c <= 'Z' ==> c in uppercase}
\end{itemize}
\clearpage
\section*{GenAI interactions}
Below you will find all interactions between the `user' (program) and the `assistant' (OpenAI).
\subsection*{Program $\rightarrow$ GenAI}
\begin{lstlisting}
You are given the following task to perform in Dafny:

Write a method that takes a string and returns only the uppercase letters in the string.

The signature should be:

method p4_3_a_extract_uppercase(s: string) returns (uppercase: string)

The method should respect the following contract:

ensures forall c :: c in uppercase ==> 'A' <= c <= 'Z', ensures forall c :: c in s && 'A' <= c <= 'Z' ==> c in uppercase

Produce and show only the Dafny body of this method, including the curly braces that surround it. Do not show the signature nor contract. You must not use 'assume' anywhere in your code.

\end{lstlisting}
\subsection*{GenAI $\rightarrow$ Program}
\textbf{System fingerprint: }
\\\textbf{ID: }chatcmpl-Co4cIu7pw8p5W55YUb5VBnYsXZj8p
\\\textbf{Model: }gpt-5.1-chat-latest
\\\textbf{Created at: }1766050182
\\\textbf{Finish reason: }stop
\\\textbf{Usage: }320 tokens in, and 283 tokens out
\begin{lstlisting}
{
  var res := "";
  var i := 0;
  while i < s.Length
    invariant 0 <= i <= s.Length
    invariant forall c :: c in res ==> 'A' <= c <= 'Z'
    invariant forall j :: 0 <= j < i && 'A' <= s[j] <= 'Z' ==> s[j] in res
  {
    if 'A' <= s[i] <= 'Z' {
      res := res + s[i..i+1];
    }
    i := i + 1;
  }
  uppercase := res;
}
\end{lstlisting}
\subsection*{Program $\rightarrow$ GenAI}
\begin{lstlisting}
When using dafny resolve, the below error is emitted and resolve fails:

problem_data/p4_3_a_extract_uppercase/program.dfy(7,14): Error: type seq<char> does not have a member Length
  |
7 |   while i < s.Length
  |               ^^^^^^

problem_data/p4_3_a_extract_uppercase/program.dfy(8,26): Error: type seq<char> does not have a member Length
  |
8 |     invariant 0 <= i <= s.Length
  |                           ^^^^^^

2 resolution/type errors detected in program.dfy

Correct the error by altering only the method body. Produce and show only the Dafny body, including the curly braces that surround it. Do not show the signature nor contract. You must not use 'assume' anywhere in your code.

\end{lstlisting}
\subsection*{GenAI $\rightarrow$ Program}
\textbf{System fingerprint: }
\\\textbf{ID: }chatcmpl-Co4cM9F2dvi7tWJgbJUsyyMJc1bJB
\\\textbf{Model: }gpt-5.1-chat-latest
\\\textbf{Created at: }1766050186
\\\textbf{Finish reason: }stop
\\\textbf{Usage: }678 tokens in, and 157 tokens out
\begin{lstlisting}
{
  var res := "";
  var i := 0;
  while i < |s|
    invariant 0 <= i <= |s|
    invariant forall c :: c in res ==> 'A' <= c <= 'Z'
    invariant forall j :: 0 <= j < i && 'A' <= s[j] <= 'Z' ==> s[j] in res
  {
    if 'A' <= s[i] <= 'Z' {
      res := res + s[i..i+1];
    }
    i := i + 1;
  }
  uppercase := res;
}
\end{lstlisting}
\section*{Final Program}
\begin{lstlisting}
method p4_3_a_extract_uppercase(s: string) returns (uppercase: string)
	ensures forall c :: c in uppercase ==> 'A' <= c <= 'Z'
	ensures forall c :: c in s && 'A' <= c <= 'Z' ==> c in uppercase
{
  var res := "";
  var i := 0;
  while i < |s|
    invariant 0 <= i <= |s|
    invariant forall c :: c in res ==> 'A' <= c <= 'Z'
    invariant forall j :: 0 <= j < i && 'A' <= s[j] <= 'Z' ==> s[j] in res
  {
    if 'A' <= s[i] <= 'Z' {
      res := res + s[i..i+1];
    }
    i := i + 1;
  }
  uppercase := res;
}
\end{lstlisting}
\section*{Total Token Usage}
\textbf{Input tokens: }998
\\\textbf{Output tokens: }440
\\\textbf{Reasoning tokens: }128
\\\textbf{Sum of `total tokens': }1438
\section*{Experiment Timings}
\textbf{Overall Experiment} started at 1766050183868, ended at 1766050192462, lasting 8594ms (8.59 seconds)
\\\textbf{Iteration \#1} started at 1766050183868, ended at 1766050188030, lasting 4162ms (4.16 seconds)
\\\textbf{Iteration \#2} started at 1766050188030, ended at 1766050192462, lasting 4432ms (4.43 seconds)
\end{document}
