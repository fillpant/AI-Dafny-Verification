\documentclass{article}
\usepackage{lmodern}
\usepackage{amsmath}
\usepackage{listings}
\usepackage{fullpage}
\usepackage{parskip}
\usepackage{xcolor}
\lstset{
	basicstyle=\ttfamily,
	columns=fullflexible,
	frame=single,
	breaklines=true,
	postbreak=\mbox{\textcolor{red}{$\hookrightarrow$}\space},
}
\begin{document}
\title{Experiment `p4\_1\_a\_sum\_even\_numbers' Results}
\author{\today}
\date{}
\maketitle
\textbf{Experiment outcome: }SUCCESS
\\\textbf{Bad responses: }0
\\\textbf{Responses containing}~\texttt{assume}~\textbf{: }0
\\\textbf{Resolution attempts: }2
\\\textbf{Hard fails (resolution): }0
\\\textbf{Soft fails (resolution): }0
\\\textbf{Verification attempts: }2
\section*{Problem Specification}
\textbf{Problem name: }p4\_1\_a\_sum\_even\_numbers
\\\textbf{Natural language statement: }Write a method with loops that computes the sum of all even numbers between 2 and 100 (inclusive).
\\\textbf{Method signature: }p4\_1\_a\_sum\_even\_numbers() returns (sum: int)
\subsection*{Ensures}
\begin{itemize}
\item \texttt{sum == 2550}
\end{itemize}
\clearpage
\section*{GenAI interactions}
Below you will find all interactions between the `user' (program) and the `assistant' (OpenAI).
\subsection*{Program $\rightarrow$ GenAI}
\begin{lstlisting}
You are given the following task to perform in Dafny:

Write a method with loops that computes the sum of all even numbers between 2 and 100 (inclusive).

The signature should be:

method p4_1_a_sum_even_numbers() returns (sum: int)

The method should respect the following contract:

ensures sum == 2550

Produce and show only the Dafny body of this method, including the curly braces that surround it. Do not show the signature nor contract. You must not use 'assume' anywhere in your code.

\end{lstlisting}
\subsection*{GenAI $\rightarrow$ Program}
\textbf{System fingerprint: }
\\\textbf{ID: }chatcmpl-Co4ZCPohmGdckmqGjaWLaPN0gR9wm
\\\textbf{Model: }gpt-5.1-chat-latest
\\\textbf{Created at: }1766049990
\\\textbf{Finish reason: }stop
\\\textbf{Usage: }260 tokens in, and 412 tokens out
\begin{lstlisting}
{
  var i := 2;
  sum := 0;
  while i <= 100
    invariant 2 <= i <= 102
    invariant sum == ((i - 2) * i) / 4
  {
    sum := sum + i;
    i := i + 2;
  }
}
\end{lstlisting}
\subsection*{Program $\rightarrow$ GenAI}
\begin{lstlisting}
When using dafny verify, the below error is emitted and verify fails:

problem_data/p4_1_a_sum_even_numbers/program.dfy(3,0): Error: a postcondition could not be proved on this return path
  |
3 | {
  | ^

problem_data/p4_1_a_sum_even_numbers/program.dfy(2,13): Related location: this is the postcondition that could not be proved
  |
2 | 	ensures sum == 2550
  |              ^^


Dafny program verifier finished with 0 verified, 1 error

Correct the error by altering only the method body. Produce and show only the Dafny body, including the curly braces that surround it. Do not show the signature nor contract. You must not use 'assume' anywhere in your code.

\end{lstlisting}
\subsection*{GenAI $\rightarrow$ Program}
\textbf{System fingerprint: }
\\\textbf{ID: }chatcmpl-Co4ZJBiNz2DYNrxZ7bILiOKguVyXB
\\\textbf{Model: }gpt-5.1-chat-latest
\\\textbf{Created at: }1766049997
\\\textbf{Finish reason: }stop
\\\textbf{Usage: }581 tokens in, and 371 tokens out
\begin{lstlisting}
{
  var i := 2;
  sum := 0;
  while i <= 100
    invariant i % 2 == 0
    invariant 2 <= i <= 102
    invariant sum == ((i - 2) / 2) * (((i - 2) / 2) + 1)
  {
    sum := sum + i;
    i := i + 2;
  }
}
\end{lstlisting}
\section*{Final Program}
\begin{lstlisting}
method p4_1_a_sum_even_numbers() returns (sum: int)
	ensures sum == 2550
{
  var i := 2;
  sum := 0;
  while i <= 100
    invariant i % 2 == 0
    invariant 2 <= i <= 102
    invariant sum == ((i - 2) / 2) * (((i - 2) / 2) + 1)
  {
    sum := sum + i;
    i := i + 2;
  }
}
\end{lstlisting}
\section*{Total Token Usage}
\textbf{Input tokens: }841
\\\textbf{Output tokens: }783
\\\textbf{Reasoning tokens: }576
\\\textbf{Sum of `total tokens': }1624
\section*{Experiment Timings}
\textbf{Overall Experiment} started at 1766049991801, ended at 1766050004904, lasting 13103ms (13.10 seconds)
\\\textbf{Iteration \#1} started at 1766049991802, ended at 1766049999076, lasting 7274ms (7.27 seconds)
\\\textbf{Iteration \#2} started at 1766049999076, ended at 1766050004904, lasting 5828ms (5.83 seconds)
\end{document}
