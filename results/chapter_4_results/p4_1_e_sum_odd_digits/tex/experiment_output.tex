\documentclass{article}
\usepackage{lmodern}
\usepackage{amsmath}
\usepackage{listings}
\usepackage{fullpage}
\usepackage{parskip}
\usepackage{xcolor}
\lstset{
	basicstyle=\ttfamily,
	columns=fullflexible,
	frame=single,
	breaklines=true,
	postbreak=\mbox{\textcolor{red}{$\hookrightarrow$}\space},
}
\begin{document}
\title{Experiment `p4\_1\_e\_sum\_odd\_digits' Results}
\author{\today}
\date{}
\maketitle
\textbf{Experiment outcome: }SUCCESS
\\\textbf{Bad responses: }0
\\\textbf{Responses containing}~\texttt{assume}~\textbf{: }0
\\\textbf{Resolution attempts: }1
\\\textbf{Hard fails (resolution): }0
\\\textbf{Soft fails (resolution): }0
\\\textbf{Verification attempts: }1
\section*{Problem Specification}
\textbf{Problem name: }p4\_1\_e\_sum\_odd\_digits
\\\textbf{Natural language statement: }Write a method with loops that computes the sum of all odd digits of an input. (For example, if the input is 32677, the sum would be 3 + 7 + 7 = 17.)
\\\textbf{Method signature: }p4\_1\_e\_sum\_odd\_digits(a: int) returns (sum: int)
\subsection*{Ensures}
\begin{itemize}
\item \texttt{sum >= 0}
\item \texttt{sum == s\_o\_d(a)}
\end{itemize}
\subsection*{Requires}
\begin{itemize}
\item \texttt{a >= 0}
\end{itemize}
\subsection*{Functional Code Given}
\begin{lstlisting}
function s_o_d(a: int): int
requires a >= 0
{
  if a == 0 then 0
  else if a % 2 == 0 then s_o_d(a / 10)
  else a % 10 + s_o_d(a / 10)
}
\end{lstlisting}
\clearpage
\section*{GenAI interactions}
Below you will find all interactions between the `user' (program) and the `assistant' (OpenAI).
\subsection*{Program $\rightarrow$ GenAI}
\begin{lstlisting}
You are given the following task to perform in Dafny:

Write a method with loops that computes the sum of all odd digits of an input. (For example, if the input is 32677, the sum would be 3 + 7 + 7 = 17.)

The signature should be:

method p4_1_e_sum_odd_digits(a: int) returns (sum: int)

The method should respect the following contract:

requires a >= 0, ensures sum >= 0, ensures sum == s_o_d(a)

The contract uses the following dafny code:

function s_o_d(a: int): int
requires a >= 0
{
  if a == 0 then 0
  else if a % 2 == 0 then s_o_d(a / 10)
  else a % 10 + s_o_d(a / 10)
}

These function(s) must not be used in your implementation of the method.

Produce and show only the Dafny body of this method, including the curly braces that surround it. Do not show the signature nor contract. You must not use 'assume' anywhere in your code.

\end{lstlisting}
\subsection*{GenAI $\rightarrow$ Program}
\textbf{System fingerprint: }
\\\textbf{ID: }chatcmpl-Co4aIgFyG5zil5Z0nV05CmFICreSc
\\\textbf{Model: }gpt-5.1-chat-latest
\\\textbf{Created at: }1766050058
\\\textbf{Finish reason: }stop
\\\textbf{Usage: }510 tokens in, and 438 tokens out
\begin{lstlisting}
{
  var x := a;
  sum := 0;
  while x > 0
    invariant x >= 0
    invariant sum >= 0
    invariant sum + s_o_d(x) == s_o_d(a)
  {
    var d := x % 10;
    if d % 2 != 0 {
      sum := sum + d;
    }
    x := x / 10;
  }
}
\end{lstlisting}
\section*{Final Program}
\begin{lstlisting}
function s_o_d(a: int): int
requires a >= 0
{
  if a == 0 then 0
  else if a % 2 == 0 then s_o_d(a / 10)
  else a % 10 + s_o_d(a / 10)
}

method p4_1_e_sum_odd_digits(a: int) returns (sum: int)
	requires a >= 0
	ensures sum >= 0
	ensures sum == s_o_d(a)
{
  var x := a;
  sum := 0;
  while x > 0
    invariant x >= 0
    invariant sum >= 0
    invariant sum + s_o_d(x) == s_o_d(a)
  {
    var d := x % 10;
    if d % 2 != 0 {
      sum := sum + d;
    }
    x := x / 10;
  }
}
\end{lstlisting}
\section*{Total Token Usage}
\textbf{Input tokens: }510
\\\textbf{Output tokens: }438
\\\textbf{Reasoning tokens: }320
\\\textbf{Sum of `total tokens': }948
\section*{Experiment Timings}
\textbf{Overall Experiment} started at 1766050060475, ended at 1766050067314, lasting 6839ms (6.84 seconds)
\\\textbf{Iteration \#1} started at 1766050060476, ended at 1766050067314, lasting 6838ms (6.84 seconds)
\end{document}
