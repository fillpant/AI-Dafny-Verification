\documentclass{article}
\usepackage{lmodern}
\usepackage{amsmath}
\usepackage{listings}
\usepackage{fullpage}
\usepackage{parskip}
\usepackage{xcolor}
\lstset{
	basicstyle=\ttfamily,
	columns=fullflexible,
	frame=single,
	breaklines=true,
	postbreak=\mbox{\textcolor{red}{$\hookrightarrow$}\space},
}
\begin{document}
\title{Experiment `2\_7\_circle\_properties' Results}
\author{\today}
\date{}
\maketitle
\textbf{Experiment outcome: }SUCCESS
\\\textbf{Bad responses: }0
\\\textbf{Responses containing}~\texttt{assume}~\textbf{: }0
\\\textbf{Resolution attempts: }1
\\\textbf{Hard fails (resolution): }0
\\\textbf{Soft fails (resolution): }0
\\\textbf{Verification attempts: }1
\section*{Problem Specification}
\textbf{Problem name: }2\_7\_circle\_properties
\\\textbf{Natural language statement: }Write a method that takes a radius and then returns the area and circumference of a circle with that radius and the volume and surface area of a sphere with that radius.
\\\textbf{Method signature: }circle\_properties(radius: real) returns (circumference: real, area: real, surface\_area: real, volume: real)
\subsection*{Ensures}
\begin{itemize}
\item \texttt{circumference == 2.0 * 3.14159 * radius}
\item \texttt{area == 3.14159 * radius * radius}
\item \texttt{surface\_area == 4.0 * 3.14159 * radius * radius}
\item \texttt{volume == (4.0 / 3.0) * 3.14159 * radius * radius * radius}
\end{itemize}
\clearpage
\section*{GenAI interactions}
Below you will find all interactions between the `user' (program) and the `assistant' (OpenAI).
\subsection*{Program $\rightarrow$ GenAI}
\begin{lstlisting}
You are given the following task to perform in Dafny:

Write a method that takes a radius and then returns the area and circumference of a circle with that radius and the volume and surface area of a sphere with that radius.

The signature should be:

method circle_properties(radius: real) returns (circumference: real, area: real, surface_area: real, volume: real)

The method should respect the following contract:

ensures circumference == 2.0 * 3.14159 * radius, ensures area == 3.14159 * radius * radius, ensures surface_area == 4.0 * 3.14159 * radius * radius, ensures volume == (4.0 / 3.0) * 3.14159 * radius * radius * radius

Produce and show only the Dafny body of this method, including the curly braces that surround it. Do not show the signature nor contract. You must not use 'assume' anywhere in your code.

\end{lstlisting}
\subsection*{GenAI $\rightarrow$ Program}
\textbf{System fingerprint: }
\\\textbf{ID: }chatcmpl-Co4PYaliEOeUDWuAc90IHeSydYUWk
\\\textbf{Model: }gpt-5.1-chat-latest
\\\textbf{Created at: }1766049392
\\\textbf{Finish reason: }stop
\\\textbf{Usage: }440 tokens in, and 290 tokens out
\begin{lstlisting}
{
  circumference := 2.0 * 3.14159 * radius;
  area := 3.14159 * radius * radius;
  surface_area := 4.0 * 3.14159 * radius * radius;
  volume := (4.0 / 3.0) * 3.14159 * radius * radius * radius;
}
\end{lstlisting}
\section*{Final Program}
\begin{lstlisting}
method circle_properties(radius: real) returns (circumference: real, area: real, surface_area: real, volume: real)
	ensures circumference == 2.0 * 3.14159 * radius
	ensures area == 3.14159 * radius * radius
	ensures surface_area == 4.0 * 3.14159 * radius * radius
	ensures volume == (4.0 / 3.0) * 3.14159 * radius * radius * radius
{
  circumference := 2.0 * 3.14159 * radius;
  area := 3.14159 * radius * radius;
  surface_area := 4.0 * 3.14159 * radius * radius;
  volume := (4.0 / 3.0) * 3.14159 * radius * radius * radius;
}
\end{lstlisting}
\section*{Total Token Usage}
\textbf{Input tokens: }440
\\\textbf{Output tokens: }290
\\\textbf{Reasoning tokens: }192
\\\textbf{Sum of `total tokens': }730
\section*{Experiment Timings}
\textbf{Overall Experiment} started at 1766049393392, ended at 1766049398916, lasting 5524ms (5.52 seconds)
\\\textbf{Iteration \#1} started at 1766049393392, ended at 1766049398916, lasting 5524ms (5.52 seconds)
\end{document}
