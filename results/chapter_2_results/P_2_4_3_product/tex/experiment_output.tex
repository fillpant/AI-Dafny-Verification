\documentclass{article}
\usepackage{lmodern}
\usepackage{amsmath}
\usepackage{listings}
\usepackage{fullpage}
\usepackage{parskip}
\usepackage{xcolor}
\lstset{
	basicstyle=\ttfamily,
	columns=fullflexible,
	frame=single,
	breaklines=true,
	postbreak=\mbox{\textcolor{red}{$\hookrightarrow$}\space},
}
\begin{document}
\title{Experiment `P\_2\_4\_3\_product' Results}
\author{\today}
\date{}
\maketitle
\textbf{Experiment outcome: }SUCCESS
\\\textbf{Bad responses: }0
\\\textbf{Responses containing}~\texttt{assume}~\textbf{: }0
\\\textbf{Resolution attempts: }1
\\\textbf{Hard fails (resolution): }0
\\\textbf{Soft fails (resolution): }0
\\\textbf{Verification attempts: }1
\section*{Problem Specification}
\textbf{Problem name: }P\_2\_4\_3\_product
\\\textbf{Natural language statement: }Write a method that takes two integers and then returns the product.
\\\textbf{Method signature: }product(x: int, y: int) returns (result:int)
\subsection*{Ensures}
\begin{itemize}
\item \texttt{result == x * y}
\end{itemize}
\clearpage
\section*{GenAI interactions}
Below you will find all interactions between the `user' (program) and the `assistant' (OpenAI).
\subsection*{Program $\rightarrow$ GenAI}
\begin{lstlisting}
You are given the following task to perform in Dafny:

Write a method that takes two integers and then returns the product.

The signature should be:

method product(x: int, y: int) returns (result:int)

The method should respect the following contract:

ensures result == x * y

Produce and show only the Dafny body of this method, including the curly braces that surround it. Do not show the signature nor contract. You must not use 'assume' anywhere in your code.

\end{lstlisting}
\subsection*{GenAI $\rightarrow$ Program}
\textbf{System fingerprint: }
\\\textbf{ID: }chatcmpl-Co4P9sOlTK1vMpzms2OEHOxKAxtsP
\\\textbf{Model: }gpt-5.1-chat-latest
\\\textbf{Created at: }1766049367
\\\textbf{Finish reason: }stop
\\\textbf{Usage: }240 tokens in, and 92 tokens out
\begin{lstlisting}
{
  result := x * y;
}
\end{lstlisting}
\section*{Final Program}
\begin{lstlisting}
method product(x: int, y: int) returns (result:int)
	ensures result == x * y
{
  result := x * y;
}
\end{lstlisting}
\section*{Total Token Usage}
\textbf{Input tokens: }240
\\\textbf{Output tokens: }92
\\\textbf{Reasoning tokens: }64
\\\textbf{Sum of `total tokens': }332
\section*{Experiment Timings}
\textbf{Overall Experiment} started at 1766049368593, ended at 1766049371893, lasting 3300ms (3.30 seconds)
\\\textbf{Iteration \#1} started at 1766049368593, ended at 1766049371893, lasting 3300ms (3.30 seconds)
\end{document}
